\documentclass[12pt,a4]{article} %[font size, tamano de hoja]{Tipo de documento}

\usepackage[left=1.8cm,right=1.8cm,top=32mm,columnsep=20pt]{geometry}

\usepackage[utf8]{inputenc} %Formato de codificación
\usepackage[spanish, es-tabla, es-nodecimaldot]{babel}
\usepackage{amsmath} %paquete para escribir ecuaciones matemáticas
\usepackage{float} %Para posicionar figuras
\usepackage{graphicx} %Para poder poner figuras
\usepackage{hyperref} %Permite usar hipervínculos 
\usepackage{multicol} %Para hacer doble columna
\usepackage[sorting=none]{biblatex} %Imports biblatex package. To cite use \cite{reference_label}
\addbibresource{tp2.bib} %Import the bibliography file



\title{Métodos Numéricos y Optimización\\

 Trabajo práctico N$^{\circ}$2\\
\vspace{20mm}

 Insertar nombre
 
}

\author{Valentino Arbelaiz y Alejo Zimmermann\\ [2mm] %\\ para nueva línea
\small Universidad de San Andrés, Buenos Aires, Argentina}
\date{1er Semestre 2024}
% Tamanos de letra: 
% \tiny	
% \scriptsize
% \footnotesize
% \small	
% \normalsize	
% \large	
% \Large	
% \LARGE	
% \huge	
% \Huge


%Todo lo que está antes de begin{document} es el preámbulo
\begin{document}
\vspace{1cm} % Ajusta la distancia vertical entre la fecha y la imagen



\maketitle
% \begin{center}
% \includegraphics[width=5cm]{logoUdesa.png} % Ajusta la ruta y el tamaño de la imagen
% \end{center}


\begin{abstract}
 
\vspace{2mm}
\end{abstract}

\begin{multicols}{2}
\raggedcolumns

\section{Introducción}

\section{Métodos\cite{burdenfaires}}

\section{Implementación}

\section{Resultados y análisis}

\section{Conclusión}


\appendix




\end{multicols}

\printbibliography



\end{document}